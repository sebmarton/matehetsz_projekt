%\documentclass{article}
\documentclass[%
 reprint,onecolumn,
%superscriptaddress,
%groupedaddress,
%unsortedaddress,
%runinaddress,
%frontmatterverbose, 
%preprint,
%preprintnumbers,
%nofootinbib,
%nobibnotes,
%bibnotes,
 amsmath,amssymb,
 aps,
%pra,
%prb,
%rmp,
%prstab,
%prstper,
%floatfix,
]{revtex4-2}
\usepackage[utf8]{inputenc}
\usepackage{amsmath}
%\usepackage[magyar]{babel}
\usepackage{graphicx}
\graphicspath{ {./images/} }


\begin{document}

\title{Hogyan mozog a Föld a Nap körül? \\
Differenciálegyenletek számítógépes megoldása 
középiskolásoknak}

\author{Sebestyén Márton}
\affiliation{Budapest-Fasori Evangélikus Gimnázium}

\author{Pályi András}
\affiliation{Elméleti Fizika Tanszék, 
Budapesti Műszaki és Gazdaságtudományi Egyetem}

\date{\today}



\maketitle

\tableofcontents

\section{Bevezetés}

Ebben a rövid jegyzetben bemutatjuk, hogy hogyan
lehet egyszerű mechanika feladatokat számítógéppel
megoldani. 
Itt azt a konkrét célt tűzzük ki, hogy néhány lépésben 
eljussunk a bolygómozgást leíró mozgásegyenletek
megoldásáig, azaz leírjuk például azt, hogy hogyan
mozog a Föld a Nap körül.
A jegyzet hasznos lehet a fizika és/vagy a 
differenciálegyenletek és/vagy a programozás 
iránt érdeklődő középiskolások vagy
egyetemisták számára.


\section{A legegyszerűbb példa: egy elsőrendű közönséges
differenciálegyenlet}

Először egy olyan feladatot veszünk, ami
nem kapcsolódik konkrét mozgáshoz, de 
vele példát tudunk mutatni
egy egyszerű \emph{differenciálegyenlet}re,
és az annak \emph{diszkretizációjá}val kapott
\emph{differenciaegyenlet}re.
Ez a diszkretizáció az az eszköz, amit a későbbiekben
például a Föld Nap körüli mozgásának leírásakor
is használni fogunk. 

Tekintsünk egy nagyon egyszerű
elsőrendű differenciálegyenletet:
\begin{equation}
\label{eq:elsorendupelda}
    y'(t) = 3 c t.
\end{equation}
Bár ehhez az egyenlethez 
nem kapcsolunk konkrét mechanikai
rendszert, a fizikai alkalmazást szem előtt
tartva 
gondoljunk úgy $y$-ra, mint egy 
egydimenzióban mozgó pontszerű
test (tömegpont) helyzetét megadó koordinátára,
azaz mértékegysége legyen méter (m).
Ebben a képben $t$ változó jelöli az időt, 
azaz mértékegysége másodperc (s).
Ezekből következik, hogy a $c$ konstans
mértékegysége $\text{m}/\text{s}^2$.

Tegyük fel, hogy ismert a tömegpont
pozíciója, $y_0 = y(t=0)$, 
a $t=0$ időpontban. 
Kérdés, hogyan változik ez a pozíció
az idő függvényeként, $y(t) = ?$

Az \eqref{eq:elsorendupelda} egyenlet
bal oldalán szereplő deriváltat közelíthetjük
egy kis $\Delta t$ időlépést választva:
\begin{equation}
    \frac{y(t+\Delta t)-y(t)}{\Delta t} = 3 c t.
\end{equation}
Átrendezve az egyenletet, fejezzük ki a függvény későbbi értékét 
a korábbi értékével:
\begin{equation}
    y(t+\Delta t) = 3 c t \Delta t + y(t)
\end{equation}
Ebből az egyenletből már látszik, hogy hogyan
lehet számítógéppel (\emph{numerikusan}) 
közelítőleg kiértékelni a tömegpont
pozícióját a $\Delta t$, $2 \Delta t$, $3\Delta t$,
... időpontokban. 

A számítási eljárást bemutatjuk a következő
konkrét példán. Legyen $c=1 \text{m}/\text{s}^2$,
és $y_0 = 4$ m. 
Rajzoljuk ki a tömegpont pozíciójának
időfüggését a $\Delta t = 0.01$ időlépés
használatával. 
Hasonlítsuk össze az így kapott görbét, és
az egzaktul, zárt alakban felírt megoldást, 
ami ilyen alakú:
\begin{equation}
    y(t) = y_0 + \frac{3}{2} c t^2
\end{equation}

\section{Szabadesés}

Az előző szakaszban egy elsőrendű közönséges
lineáris differenciálegyenletet vizsgáltunk. 
Most egy másodrendű közönséges differenciálegyenlet
következik, ami a testek szabadesését írja le. 


\begin{figure}[h]
\includegraphics[width=10cm]{images/szabadeses.png}
\caption{Szabadesés $v_0$ kezdősebességgel, $z_0$ kezdeti
magasságból.\label{fig:szabadeses}}
\end{figure}

A szabadesés problémáját és a kezdeti feltételeket
a FIG.~\ref{fig:szabadeses}. brán mutatjuk be.
Függőleges irányban áll a z tengely,
felfelé irányítva. 
A mozgásegyenlet nem más, mint Newton II. törvénye:
\begin{equation}
    m \ddot{z}(t) = - m g,
\end{equation}
ahol $m$ a test tömege, és $g$ a nehézségi 
gyorsulás. 
A mozgás leírásához meg kell adni 
két kezdeti feltételt: 
kezdeti pozíció, $z_0 = z(t=0)$, 
és 
kezdeti sebesség, $v_0 = \dot{z}(t=0)$.
Ezek ismeretében a feladat általában 
a mozgás leírása, azaz a pozíció és a 
sebesség időbeli változásának 
kiszámítása, $z(t)= ? $ és $v(t) = ?$.
Ennek a problémának az egzakt megoldása természetesen 
jól ismert, így ezzel az ismert megoldással össze tudjuk 
majd vetni a numerikus megoldásunkat, így igazolva
a numerikus módszer helyességét. 


A számítógépes megoldáshoz szükségünk van 
a második derivált differenciál-közelítésére.
Ennek egy lehetséges felírása a következőképpen
adható meg. 

Az első derivált a $t+\Delta t/2$ helyen 
közelítőleg kifejezhető így:
\begin{equation}
   \dot{z}(t+\Delta t/2)
   \approx \frac{z(t+\Delta t) - z(t)}{\Delta t}
\end{equation}
Másrészt, az 
első derivált a $t-\Delta t/2$ helyen 
közelítőleg kifejezhető így:
\begin{equation}
    \dot{z}(t-\Delta t/2) \approx 
    \frac{z(t) - z(t-\Delta t)}{\Delta t}
\end{equation}
A második derivált közelítőleg kifejezhető 
az első deriválttal: 
\begin{equation}
\ddot{z}(t)
\approx 
\frac{\dot{z}(t+\Delta t/2) - \dot{z}(t-\Delta t/2)}{\Delta t}
\end{equation}
Mindezekből következik, hogy 
a második derivált
közelítőleg kifejezhető így:
\begin{equation}
\ddot{z}(t)
\approx 
  \frac{\frac{z(t+\Delta t) - z(t)}{\Delta t}
  - \frac{z(t) - z(t-\Delta t)}{\Delta t}}{\Delta t}
\end{equation}



    
Tehát a fenti mozgásegyenlet 
differenciaegyenlet formájában így írható fel:
\begin{equation}
\label{eq:differenciaegyenlet}
    m \frac{\frac{z(t+\Delta t) - z(t)}{\Delta t}
  - \frac{z(t) - z(t-\Delta t)}{\Delta t}}{\Delta t}
    = - m g.
\end{equation}    
Ennek átrendezésével pedig kapjuk a következő 
formulát, amit már közvetlenül használhatunk 
a számítógépes programunkban:
ha ismerjük a pozíciót a $t-\Delta t$ időpontban,
és a $t$ időpontban, akkor a $t+\Delta t$ 
időpontban ez lesz a pozíció:
\begin{equation}
\label{eq:zkesobbi}
    z(t+\Delta t) = -g \Delta t^2 + 2 z(t) - z(t-\Delta t).
\end{equation}
    
A kezdeti feltételek közül a $z_0$ kezdőpozíció
adja a $z(t=0)$ értéket, míg a $v_0$
kezdősebességből megkaphatjuk az első
időlépés utáni pozíciót, $z(\Delta t) = z_0 + v_0 \Delta t$.
Ezt a két értéket beírva \eqref{eq:zkesobbi} jobb oldalába,
a bal oldal megadja $z(2\Delta t)$ értékét, majd
ezt a módszert iterálva az összes 
későbbi $n\Delta t$ időpontban is kiszámíthatjuk a
pozíció értékét. 
Ha a sebességértékekre is szükségünk van, akkor azt
például a 
\begin{equation}
    v(n \Delta t) \approx \frac{z((n+1) \Delta t) - z(n\Delta t)}{\Delta t}
\end{equation}
egyenletből kaphatjuk meg.


    
A numerikus megoldásunkat könnyen ellenőrizhetjük, hiszen
a szabadesés problémájára ismerjük az egzakt
megoldást:
\begin{equation}
    z(t) = z_0 + v_0 t - \frac{1}{2} gt^2
\end{equation}





Egy konkrét feladat a következő. 
Írjuk le a szabadesést, azaz a pozíció időfüggését,
a következő kezdeti feltételek esetén:
$z(t=0) = z_0 = 1$ m, 
$v_0 = 0 \frac{m}{s}$.
Kérdés: $z(t) = ?$ 0 s és 5 s közötti időablakban.

A fent leírt numerikus eljárással kapott eredményt
mutatják a kék adatpontok a FIG.~\ref{fig:szabadeseseredmeny}
ábrán, egy konkrét $\Delta t$ időlépés-érték esetén.
Összehasonlításképpen a fekete vonal mutatja
az egzakt megoldást;
a hasonlóság meggyőző.

% Procedúra:
% \begin{enumerate}
%     \item 12-es egyenletből kifejezni a $z(\Delta t)$
%     értékét
%     \item ezt beírni a 10-es egyenletbe, úgy hogy 
%     azt a $t=\Delta t$ esetben tekintjük
%     \item 10-es egyenlet kiértékelése megadja
%     a $z(2\Delta t)$ értékét.
%     \item ezután a 10-es egyenletet kell ismételgetni
%     úgy, hogy a bal oldalon $z(3\Delta t)$ szerepel, 
%     majd $z(4\Delta t)$, és így tovább.
% \end{enumerate}

\begin{figure}[h]
    \centering
    \includegraphics[width = 10cm]{images/szabadeses_plot.png}
    \caption{A szabadesés numerikus szimulációja. 
    Kék: Szimulált adatok, időlépés: $\Delta t = 0.02$ s.
    Fekete: egzakt megoldás. \\
    \bf{Forráskód: code/szabadeses}
    \label{fig:szabadeseseredmeny}}
\end{figure}






\section{Rezgőmozgás két dimenzióban}





Az előző szakaszokban egydimenziós mozgást vizsgáltunk, 
ebben a szakaszban viszont továbblépünk a 
kétdimenziós mozgások leírására.
Egy kétdimenziós rezgőmozgás egyenleteit fogjuk megoldani 
numerikusan. 
Az egzakt megoldás itt is ismert. 


% Egy test körszerű pályán halad, melynek pozícióvektora $\vec{r}$.
% Fejezzük ki a centripetális erőt Newton II. törvényéből:
% \begin{equation}
%     F_{cp} = m a \Rightarrow a = \frac{v^{2}}{r} 
% \end{equation}

% Fejezzük ki a centripetális erőt az $\vec{r}$ vektor és egy $\alpha$ konstans szorzataként.

Feltételezzük, hogy a síkon mozgó tömegpontra egy centrális
erő hat, ami a sík origója felé húzza a tömegpontot, 
mégpedig az origótól vett távolsággal arányos nagyságú
erővel. 
Formálisan tehát az erővektor és a pozícióvektor így függ össze:
% Mivel a centripetális erő ellentétes irányú az $\vec{r}$ vektorral, negatív előjlelet kap:
\begin{equation}
    \vec{F} = - \alpha \vec{r}(t)
\end{equation}
Ez az egyenlet maga a mozgásegyenlet is.  
Az elrendezés a FIG.~\ref{fig:rezgomozgas} ábrán látható.
A feladat akkor válik jól definiálttá, ha megadjuk a kezdeti 
pozícióvektort és sebességvektort is. 

\begin{figure}[h]
    \centering
    \includegraphics[width = 10cm]{images/korm.png}
    \caption{Kétdimenziós rezgőmozgás.
    \label{fig:rezgomozgas}}
\end{figure}


A vektorokkal felírt fenti mozgásegyenlet diszkretizációja előtt 
írjuk ki a vektorokat komponensenként. 
Például a helyvektor 
\begin{equation}
    \vec{r} = (x(t), y(t))
\end{equation}
alakú.
Írjuk fel a differenciálegyenletet mindkét koordinátára:
\begin{equation}
    m \ddot{x}(t) = - \alpha x(t)
\end{equation}
\begin{equation}
    m \ddot{y}(t) = - \alpha y(t)    
\end{equation}
A feladatunkat jelentősen leegyszerűsíti, hogy az egyik egyenlet
csak $x$-et, a másik csak $y$-t tartalmazza,
azaz a két egyenlet \emph{szétcsatolódik}.


A fenti differenciálegyenletek diszkretizációjával 
ezeket a differencia-egyenleteket kapjuk:
\begin{eqnarray}
    v_{x}(t + \Delta t) &=& v_x(t) - \beta  x(t) \Delta t, \\
    x(t + \Delta t) &=&  x(t) + v_{x}(t) \Delta t, \\ 
    v_{y}(t + \Delta t) &=& v_y(t) - \beta  y(t) \Delta t, \\
    y(t + \Delta t) &=& y(t) + v_{y}(t) \Delta t, 
\end{eqnarray}
ahol $\beta = \alpha / m$.

Mivel másodrendű a differenciálegyenletünk, és két koordinátánk van, négy kezdőfeltételt kell megadnunk, és ezen kívül az $\alpha$ konstanst is.
Tegyük konkréttá a feladatot!
Oldjuk meg numerikusan a fenti differenciálegyenletet 
$\Delta{t} = 0.01$ s időlépéssel, 
$\beta = 7 \, \text{N}/ (\text{m}\, \text{kg}) = 7 /\text{s}^2$ 
együtthatóval, 
és a következő kezdeti paraméterekkel:
\begin{enumerate}
    \item $x_{0} = 1$ m,
    \item $y_{0} = 0$ m,
    \item $v_{x0} = 0$ m/s,
    \item $v_{y0} = 1$ m/s.
\end{enumerate}

\begin{figure}[h]
    \centering
    \includegraphics[width = 10cm]{images/kormozgas_plot.png}
    \caption{A körmozgás numerikus szimulációja. Kék: x koordináta.
    Piros: y koordináta. Időlépés: $\Delta t = 0.01$ s. \\
    \bf{Forráskód: code/kormozgas}
    \label{fig:rezgomozgaseredmeny}}
\end{figure}

A numerikus megoldást a FIG.~\ref{fig:rezgomozgaseredmeny} mutatja. 
Egy koszinusz- (kék) és egy szinuszfüggvényt (piros) látunk, 
ugyanazzal a periódusidővel
($T = 2\pi / \omega \approx 2.37 \, \text{s}$, 
ahol $\omega = \sqrt{\beta} \approx 2.64$),
de különböző amplitúdóval. 







\section{A Föld mozgása a Nap körül}


\begin{figure}[h]
    \centering
    \includegraphics[width = 10cm]{images/grav.png}
    \caption{A Föld mozgása a Nap körül.}
    \label{fig:napfold}
\end{figure}

Végül megvizsgáljuk a Nap gravitációs terében lévő Föld mozgását. 
Feltevésünk szerint a Nap az origóban van, és rögzített, 
lásd FIG.~\ref{fig:napfold} ábra. 
A Földnek a Naphoz viszonyított pozícióvektorát jelöljük $\vec{r}$-el,
és feltesszük hogy a Föld mozgása az xy síkban történik.

A mozgásegyenletet úgy kapjuk, hogy felírjuk a Földre a Nap által
kifejtett gravitációs erővektort:
\begin{equation}
    \vec{F_{g}} = \ddot{\vec{r}} = - \hat{\vec{r}} \gamma \frac{m_{N}m_{F}}{r^{2}}
\end{equation}
ahol $\hat{\vec{r}} = \vec{r}/r$ az $\vec{r}$ irányú egységvektor,
és $\gamma = 6.67 \cdot 10^{-11} \frac{Nm^2}{kg^2}$ a gravitációs állandó.

Fejezzük ki az $\hat{\vec{r}}$ helyvektort:
\begin{equation}
    \hat{\vec{r}} = \frac{\vec{r}}{|\vec{r}|} = \frac{(x, y)}{\sqrt{x^{2} + y^{2}}}.
\end{equation}
Ebből kapjuk, hogy
\begin{equation}
    \vec{F_{g}} = - \gamma \frac{{m_{F} m_{N} \vec{r}}}{r^{3}},
\end{equation}
illetve ezt a mozgásegyenletet koordinátánként kiírva kapjuk, hogy 
\begin{eqnarray}
    \ddot{x}(t) &=& - \gamma \frac{m_{N} x(t)}{(x^{2} + y^{2})^{\frac{3}{2}}}
  \\
    \ddot{y}(t) &=& - \gamma \frac{m_{N} y(t)}{(x^{2} + y^{2})^{\frac{3}{2}}}
\end{eqnarray}

Az ezekből diszkretizáció révén kapható 
differenciaegyenletek:
\begin{eqnarray}
    v_{x}(t + \Delta t) &=& - \gamma \frac{m_{N} x(t)}{(x^2 + y^2)^\frac{3}{2}} \Delta t \\
    x(t + \Delta t) &=& x(t) + v_{x}(t) \Delta t \\
    v_{y}(t + \Delta t) &=& - \gamma \frac{m_{N} y(t)}{(x^2 + y^2)^\frac{3}{2}} \Delta t \\
    y(t + \Delta t) &=& y(t) + v_{y}(t) \Delta t
\end{eqnarray}

Ezen egyenletek numerikus megoldását ábrázoljuk a 
FIG.~\ref{fig:ottmarad},
\ref{fig:elhagyja}, \ref{fig:beleesik} ábrákon, 
három különböző kezdeti feltétel esetén (ld. ábrafeliratok).
A FIG.~\ref{fig:ottmarad} ábrán a kezdősebesség úgy van beállítva,
hogy a Föld - a tényleges mozgásához hasonlóan - ellipszispályán
mozog a Nap körül.
A FIG.~\ref{fig:elhagyja} ábrán a kezdősebesség nagyobb, 
ezért a Föld a mozgása során eltávolodik a Naptól. 
A FIG.~\ref{fig:beleesik} ábrán a kezdősebesség kisebb, 
ezért a Föld a mozgása során beleesik a Napba. \\
{\bf Forráskód: code/orbsim}


\begin{figure}[!h]
    \centering
    \includegraphics[scale = 0.5]{images/palya1.png}
    \caption{A Föld körpályán(x = $0$ km, y = $150,000,000 km$, v = $262.5 km/s$, Időlépték: $\Delta t = 5000 s$)}
    \label{fig:ottmarad}
\end{figure}

\begin{figure}[!h]
    \centering
    \includegraphics[scale = 0.5]{images/palya2.png}
    \caption{A Föld elhagyja a Napot(x = $-225,000,000 km$, y = $150,000,000 km$, $v = 375 km/s$, Időlépték $\Delta t = 5000 s$)}
    \label{fig:elhagyja}
\end{figure}

\begin{figure}[!h]
    \centering
    \includegraphics[scale = 0.5]{images/palya3.png}
    \caption{A Föld és a Nap összeütközik(x = $-150,000,000 km$, y = $150,000,000 km$, v = $53,25 km/s$, Időlépték $\Delta t = 5000 s$)}
    \label{fig:beleesik}
\end{figure}


\section{Összefoglalás}

Ebben a jegyzetben bemutattuk, hogy hogyan
lehet a tömegpont-mechanika feladatait számítógéppel, 
numerikusan megoldani. 
Néhány lépésben eljutottunk 
a bolygómozgást leíró mozgásegyenletek
megoldásáig.

A jegyzet hasznos lehet a fizika és/vagy a 
differenciálegyenletek és/vagy a programozás 
iránt érdeklődő középiskolások vagy
egyetemisták számára.

A jegyzet a Matehetsz Mentor Programban készült, 
2020-2021-ben. 
SM készítette a numerikus számításokat, ábrákat, illetve a szöveg 
első verzióját. PA javasolta a témát és szerkesztette a szöveget. 


\end{document}



á
